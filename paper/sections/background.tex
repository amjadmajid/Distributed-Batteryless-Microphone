Here a minimal amount of background information is presented to facilitate clarifying the technical sections of the paper. 

\subsection{Energy-harvesting Devices}

%1. why \textit{small form factors EH sensor nodes}\\
Small sensors are less intrusive devices, and therefore, many applications prefer them on the bigger ones 
% (imagine the difference in the implications of embedding a sensor of the size of 2 AAA batteries and a sensor of a few cubic millimeters in volume in a shoe for step counting).  
%
Long sensors lifetime is also desirable. However, long sensors lifetime and small form factors are conflict goals.
% 2. classify EH with continuous power and tiny EH with intermittent power \\
For example, rechargeable batteries paired with energy harvesters can continuously power sensor nodes for relatively long time. But, rechargeable batteries inherent normal batteries drawbacks including increasing the size and limiting the sensor lifetime (although much longer) as rechargeable batteries typically wear out after a few hundreds charging cycles~\cite{}.
%
However, if an application's requirements put hard constraints on the size of the sensors, then removing the batteries is one of the first options to be considered. 
Battery-less energy-harvesting sensors operate intermittently. They charge a small capacitor to ensure uninterrupted operations for a minimum certain duration. Once, the capacitor has been depleted, the sensor powers down, letting the energy-harvester to accumulate energy again. 
%
Intermittent operation raises many challenges such as how to enable applications to span their execution over power failure~\cite{}, and how to enable timeliness operations when the durations of the device power-downs are indeterminate~\cite{}.
%
Big capacitors may allow longer operational periods of time, but they also need more time to charge. 
In order for a capacitor to charge, the input voltage must be higher than the accumulated voltage in the capacitor. 
This phenomena makes charging big capacitors using tiny energy harvester less efficient~\cite{}. 
boosting ambient energy using an energy-conditioning circuit is possible on the expense of device complexity, form factor, energy consumption, and cost. 

\subsection {Speech types}
%
Speech recognition algorithms can be classified based on the type of speech that they can recognize into \textit{spontaneous speech, continuous speech, connected word,} and \textit{isolated word}~\cite{gaikwad2010review}.
Systems with \textit{continuous} or \textit{spontaneous speech} recognition are the closest to natural speech, but are the most difficult to create because they need special methods to detect words boundaries~\cite{gaikwad2010review}. This is less the case for the \textit{connected word} type, where a minimum pause between the words is required.
 The type with the least complexity is the \textit{isolated word} type. It requires a period of silence on both sides of a spoken word and accepts only single words. 
 
Voice is a natural way for the human to interact with small devices. However,
implementing speech recognition on resources---memory, computation power, and energy---limited platforms is challenging, to say the least. Therefore, we attempt to recognize, with our command recognizer prototype, the simplest type of speech, isolated words. 