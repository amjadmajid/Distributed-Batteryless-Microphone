
\begin{figure}
	\centering
	\begin{subfigure}{\columnwidth}
		\includegraphics[width=\columnwidth]{figures/intermittent_problem}
		\caption{The on-time percentage of an intermittent device powered by different energy sources relative to its on/off cycle length}
		\label{fig:disInterSys}
	\end{subfigure}
	\begin{subfigure}{\columnwidth}
		\includegraphics[width=\columnwidth]{figures/PowerCycleIntermittentSystem}
		\caption{The power cycle of an intermittently powered device.}
		\label{fig:powerCycle}
	\end{subfigure}
\end{figure}
%Intermittent software enables energy harvesting devices to run reliably despite arbitrarily timed power failures.  However, intermittent devices failed to  
%
%Self-powered tiny devices promise humanity with intelligent objects. Low cost, small size, and long operational time enable them to fundamentally change healthcare~\cite{ChenG.ISSCC2011}, water infrastructure~\cite{stoianov2007pipenet}, smart building~\cite{ardakanian2016buildsys,debruin2013monjolo}, and human-natural interaction~\cite{debruin2013monjolo}. \\
%%
%They operate by harvesting energy from ambient sources such as light~\cite{margolies_tosn_2016,margolies_infocom_2016}, vibration~\cite{gorlatova_sigmetrics_2014}, and radio frequency~\cite{rf_powered_computing_gollakota_2014} and able to compute, sense, actuate, and communicate~\cite{moo, naderiparizi_rfid_2015,flickersensys2017,smith_ubicomp_2006}.Ambient power, however, is marginal and unpredictable. Therefore,
%the execution is intermittent: it is triggered when a threshold amount of energy is buffered and terminated when the energy buffer is depleted. 

%%% Background, known information 

%An intermittently powered device (or node) starts by harvesting ambient energy. Once the harvested energy reaches a threshold, software execution begins and the energy in the buffer depletes gradually until the device shutdowns. This power cycle repeats indefinitely (Figure\,\ref{fig:powerCycle}). 

Intermittently powered devices use their environment as an energy source instead of batteries. Therefore, they promise small, cheap, and maintenance-free version of the current Internet of Things (IoT) edge devices. Driven by this vision, recent years have paid significant attention to intermittent systems~\cite{hicks2017clank,lucia2017intermittent,colin2016chain,colin2018termination,yildirim2018ink}. 
%
%%% knowledge gap, unknown information
However, their inherent sporadic operation patterns (Figure~\ref{fig:disInterSys}) have prevented researchers from demonstrating a real word applications, for example, \cite{colin2016chain,hester2017timely} present activity recognition applications without driving a real sensor to capture external signals.

%
%%% Hypothesis, question, purpose statement
This paper responds to the challenge of intermittent power supply by introducing the concept of \textit{distributed intermittent systems}. A distributed intermittent system is defined as the abstraction of a group of tiny intermittently powered devices (or nodes). The on-time of the distributed intermittent system should approach continuous time as the number of intermittent devices increases. However, the on-time and off-time of  distributed intermittent systems depend on the environment and the load. As such, we do not expect, for example, a linear relationship between the number of nodes and the overall on-time.

Controlling the on/off cycle of intermittent devices enables adapting them to many real world applications. For example, once a certain on/off cycle is preserved, an intermittent wake-up receiver can be implemented; intermittent acoustic monitoring system for monitoring engines modules---the sound produced by a deformed gear tooth---can be made. Moreover, with the advances in passive communication (such as passive light~\cite{}, and backscatter tag-to-tag~\cite{} communication) battery-free miniaturized sensors can form self-powered wireless sensor network to, for instance, create smart wallpaper and revolutionize smart buildings. 


This paper pushes the boundaries of intermittent systems by:
\todo{update the contributions and challenges}
\begin{itemize}
		\item introducing \textit{distributed intermittent systems} to control the duration of the up time of intermittent sensors and increase their responsiveness,
		\item investigating the relation between distributed intermittent systems power cycle and their environment,
		\item demonstrating the \textit{world first} distributed intermittent system: a distributed microphone. 
\end{itemize}
%\todo{add the challenges and contributions}
%%% Approach, plan of attack, proposed solution



