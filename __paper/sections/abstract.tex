The main obstacles to achieve truly ubiquitous sensing are (i) the limitations of battery technology - batteries are short-lived, hazardous, bulky, and costly - and (ii) the unpredictability of ambient power. The latter causes sensors to operate intermittently, violating the availability requirements of many real-world applications. 

In this paper, we present the \textit{\fullcis} (\cis), an
intermittently-powered ``sensor'' that senses continuously! Although
a single node will frequently be off charging, a group of nodes can
--in principle-- sense 24/7 provided that their awake times are spread
apart. As communication is too expensive, we rely on inherent component
variations that induce small differences in power cycles. This basic
assumption has been verified through measurements of different nodes
and power sources. However, desynchronizing nodes is not enough.

An important finding is that a \cis designed for certain (minimal)
energy conditions will become synchronized when the available energy
exceeds the design point. Nodes employing a sleep mode (to extend
their availability) do wake up collectively at some event, process it,
and return to charging as the remaining energy is typically too low to
handle another event. This results in multiple responses (bad)
and missing subsequent events (worse) due to the synchronized charging.
To counter this undesired behavior we designed an algorithm to estimate the number of active neighbors and respond proportionally to an event. 
We show that when intermittent nodes randomize their responses to events, in favorable energy conditions, the \cis reduces the duplicated captured events by 50\% and increases the percentage of capturing entire bursts above 85\%. 
